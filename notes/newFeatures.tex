\emph{Usability features}

\begin{itemize}
\item
  \textbf{Sand Trees} - ADTool have now possibility to
  handle/create/save Sequential Attack Trees.
\item
  \textbf{Copy Paste} - in the new version of ADTool now it is possible
  to copy-paste subtrees in tree view and valuations in the domain view.
  It is also possible to copy all valuations for a given tree in the
  Valuations View.
\item
  \textbf{Multiple Document Interface} - Added possibility to open
  multiple trees at the same time. This allows i. e. to copy some
  subtree from two different trees.
\item
  \textbf{Ranking} - added possibility to list attacks ranked on the
  value of the attack
\item
  \textbf{New GUI library} - new ADTool now uses Docking Frames library.
  Old library used before does not support Mac OS for the new version of
  Java
\item
  \textbf{Saving Layout} - on exit the layout and content of the windows
  are automatically saved and restored on the next starting of ADTool
\item
  \textbf{Command Line} - now it is possible to script ADTool using
  command line. ADTool is able to open XML and save it in all supported
  formats with various options i. e. exporting best N attacks or domain
  with calculated values.
\item
  \textbf{Shifting Nodes} - user now can change the order of children
  using shift key with arrows keys
\item
  \textbf{Accepting non-strict XML} - files that are not conforming to
  XML schema but ADTool still can understand
\end{itemize}

\emph{Other features}

\begin{itemize}
\item
  new faster parser for ADTerms
\item
  rewritten the code so that both tree and ADTerms are represented by a
  single structure and programm uses less memory
\end{itemize}
